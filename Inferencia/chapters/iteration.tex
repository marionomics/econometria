\chapter{La única forma de crear valor es iterando a partir de los datos}

\begin{quote}
\textit{Failure isn’t a necessary evil. In fact, it isn’t evil at all. It is a necessary consequence of doing something new.}\\
-- Ed Catmull
\end{quote}


En 2011, durante mi primer semestre en la maestría en Economía, me decidí inscribir a un evento: Startup Weekend.

El evento empezaba un viernes por la tarde. El reto era crear un proyecto de negocio durante el fin de semana y presentarlo en Domingo. Era la primera vez que hacía algo así junto a personas tan talentosas: habían programadores, diseñadores y con perfil de negocios, cada persona más extraordinaria que la siguiente.

En la pared colgaba una lona que decía “No talk, all action”.

Me quedé enganchado con esa idea. La filosofía detrás de estas cuatro palabras era no pasar demasiado tiempo planeando y comenzar a ejecutar lo antes posible. La mejor forma de saber si una idea funciona es poniéndola a prueba.

\section{Las metodologías de negocios basadas en experimentos}
2008 fue un año muy intenso en la historia de la humanidad.

Lo recuerdo bien porque fue el año en el que salí de la universidad. En mis últimos semestres tomaba una clase donde el profesor nos explicaba en tiempo real lo que estaba pasando en los mercados financieros y cómo el mundo caía en una inminente recesión. Todo eso mientras me preparaba para salir de mi licenciatura.

Salí a un mundo muy diferente al que existía mientras estudié la licenciatura.

Ese mismo año, Eric Ries publicó el libro ``The \gls{lean-startup}''. Era una filosofía diferente a la creación de negocios, que permitía a las empresas adaptarse en tiempos cambiantes, no importa su tamaño. Su adopción fue inmediata en el mundo de las startups de tecnología.

¿En qué consiste esta metodología?

La idea central es \textbf{fracasa rápido, fracasa barato}. En otras palabras, comienza tu negocio asumiendo que en tu primera versión vas a fracasar. El producto no es lo que el cliente buscaba, te equivocaste en el canal para promocionarlo, el nicho de mercado no era en mejor. Todas estas cosas pueden y van a pasar. Tu trabajo es identificarlas al menor costo posible y cambiar rápidamente para encontrar el mejor ajuste entre producto y mercado.

Y nadie mejor para enseñarnos cómo se hace que un monje inglés del siglo XVIII.

\section{Bayes y las bolas de Billar}
Thomas Bayes era un ministro presbiteriano, mejor conocido por el teorema que lleva su nombre.

En 1763, Bayes publicó un ensayo donde explicaba una forma diferente de hacer estadística. Su impacto fue tan grande que hay una clara distinción entre la estadística frecuentista de la Bayesiana. Mientras la estadística frecuentista interpreta la probabilidad como la frecuencia a largo plazo de un evento, la bayesiana la entiende como una medida de la creencia sobre un evento.

Imaginemos un juego de billar.

Yo le pego a la bola blanca mientras tú no estás viendo. La bola se mueve por toda la mesa hasta que se detiene en algún punto aleatorio. Mido la distancia de la orilla de la mesa a donde cayó la pelota, registro lo que medí y guardo la bola blanca.

La posición de la bola está oculta para ti, pero te voy a dar algunas pistas para que la adivines.

Ahora lancemos bolas rojas. Cada bola que lancemos va a caer en un lugar aleatorio de la mesa. Yo sólo te voy a decir si cayó a la derecha o a la izquierda de la bola que lanzamos al inicio.

¿Podrías decirme en dónde cayó la pelota?
\subsection{Empieza con un conocimiento previo, pero actualízalo sobre la marcha}

Nuestra hipótesis es que existe un modelo de negocios que se ajusta de manera ideal a un mercado existente.

El problema es que no hay forma de saberlo antes de hacer la prueba. No hay libro, clase o mentoría que te de el secreto para crear un negocio innovador que genere ganancias millonarias. Si fuera conocimiento público, alguien más ya lo habría hecho y la oportunidad desaparecería.

Se llama la \gls{hipotesis-mercados-eficientes}.

Cuando el panorama es incierto, lo mejor que podemos hacer es experimentar. Esto implica tomar los pasos del método científico y aplicarlos en todos los aspectos de nuestro modelo de negocio. Todo está sujeto a modificación, si los datos nos dice que no está funcionando.

El objetivo es lograr esto al menor costo posible.

\subsection{\gls{mvp}: Mínimo Producto Viable}
Fracasa rápido, fracasa barato.

Un día estás platicando con tus amigos y decides aliarte con Dani para abrir una heladería. Platican por horas sobre los sabores que van a ofrecer, en dónde se van a establecer y el nombre que le van a poner. Incluso diseñaron un logo en una servilleta.

Sólo hay un problema: no tienen dinero para hacer todo eso.

La mayoría de las personas piensa que, si su idea es muy buena, podrían ir con un inversionista, mostrarle el plan de negocios y levantar el capital que necesitan. Ahora sólo queda implementar el plan y sentarse en un camastro en la playa a ver cómo la cuenta de banco se incrementa. Listo, eres millonario.

Excepto que eso nunca ha sucedido en la historia.

La realidad es que no hay plan de negocios que sobreviva a la realidad. No hay forma de tener todo listo. No importa si eres el mejor estratega de negocios del mundo, en algún momento tienes que hacer pruebas y fallar. No hay forma de evitarlo.

El truco es fallar de forma inteligente.

En nuestra heladería, nuestro sueño podría ser un local comercial en el centro comercial más grande de la ciudad. ¿Cuál sería la versión más ligera de este modelo de negocio? Depende de lo que deseamos comprobar.

Si deseas comprobar si hay demanda por los sabores, ¿de verdad necesitamos ese local?

Un mínimo producto viable es la versión mínima con la que puedes comprobar tu hipótesis. Por ejemplo, puedes comenzar vendiendo los helados en línea. De esta manera puedes observar cuáles son los sabores que les gustan y cuáles no.

Solo hay un detalle importante: para saber lo que funciona y lo que no, hay que usar los principios de la \textbf{inferencia causal}.

\section{Métricas y experimentos}
Todo en tu modelo de negocios es sujeto a ser parte de un experimento.

En una startup, el lanzamiento del producto se hace de manera continua. En cada versión de tu modelo de negocio hay pequeñas variaciones que hacer. Cada variación implica una comparación de las métricas relevantes.

Estos son los elementos de tu modelo de negocio de los que puedes hacer variaciones relevantes, en términos generales.
\begin{itemize}
    \item El segmento de mercado al que te diriges.
    \item La localización.
    \item Los canales por los que te publicitas.
    \item La estrategia de precios.
    \item Todas las características del producto o servicio.
\end{itemize}

Todos estos aspectos pueden afectar tus ventas, la operación de tu empresa y tu rentabilidad.

\subsection{Cuidado con las métricas de vanidad}
En un mundo lleno de datos, es fácil perdernos entre lo que es fácil y lo que vale la pena medir.
\begin{itemize}
    \item Es fácil medir los followers, likes y reposts en nuestra página de redes sociales.
    \item Es fácil dar seguimiento al número de suscriptores de nuestro newsletter o al tráfico de la página.
\end{itemize}
Pero esas son métricas que no nos dicen mucho sobre el rendimiento real de nuestro negocio. A estas métricas les llamamos \gls{vanity-metric}. Las \textit{métricas de vanidad} se sienten bien, pero seguirlas no ayuda a hacer mejoras sustanciales en el negocio.

Por el contrario, deseas que tus métricas sean \textit{accionables}.

Algunos ejemplos de Métricas accionables:
\begin{itemize}
    \item Tasas de conversión
    \item Tasas de opt-in donde los usuarios pongan su correo en una landing page.
    \item Costo de adquisición de clientes (\gls{cac}).
\end{itemize}
No siempre es fácil seguir estas métricas.

En ocasiones es necesario hacer fórmulas complejas para darles seguimiento. En otros casos requiere de hacer registros cuidadosos que toman tiempo y recursos adicionales. Pero estas son las métricas que vale la pena seguir.

Si no estás siguiendo estas métricas, estás dejando dinero en la mesa.

\section{Realiza cambios basados en la evidencia}
Hasta el momento hemos:
\begin{itemize}
    \item Lanzado un producto mínimo viable.
    \item Definiste tus métricas, evitando las métricas de vanidad.
\end{itemize}
Es momento de divertirnos.

El juego se llama “usa el método científico para definir tu negocio”. La idea es hacer modificaciones continuas a tu modelo de negocio de acuerdo a lo que la evidencia te dice.

Si la evidencia dice que los helados en cono son mejores que en vaso, elige helados en cono. Si la evidencia dice que los sabores tradicionales se venden más que los exóticos, pero los exóticos se venden más caro, puedes elegir enfocar tus sabores de acuerdo al cliente al que quieres atender.

No importa lo que digan los libros, tu tía o los “expertos en mercadotecnia”. Lo único que importa es lo que digan los datos.

\section{Cómo hacer una empresa basada en datos}
Esta metodología no sólo sirve para startups o empresas de tecnología, también se puede implementar en empresas en marcha.

Lo importante es que se sigan los siguientes principios:
\begin{itemize}
    \item Lanzamiento continuo: En ningún momento estaremos en la versión definitiva de la empresa. Siempre hay algo que modificar.
    \item Fracasa rápido, fracasa barato: Si quieres probar una hipótesis, hazlo de la manera mas mínima posible y con iteraciones cortas.
\end{itemize}
Así se vería en nuestro emprendimiento de helados.
\begin{itemize}
    \item \textbf{Paso \#1: Identifica tu métrica de interés}. Por ejemplo, imagina que quieres aumentar las ventas de helados en tu empresa en invierno. ¿Es posible?
    \item \textbf{Paso \#2: Formula tu hipótesis}. ¿Qué variable crees que es clave para hacer crecer las ventas? Sabemos que es un mito que comer helado en invierno causa resfriados. Si viviéramos en un mundo de \textit{homo economicus}, a nadie le importaría comer helado en invierno, pero si tendrían problemas con hacerlo en un lugar cerrado, porque circulan los virus (¡espero que hayamos aprendido algo con la pandemia!). ¿Qué pasaría si vendiéramos helados con probióticos durante el invierno?
    \item \textbf{Paso \#3: Identifica el mínimo viable}. Lo peor que podrías hacer es crear un nueva línea de productos y hacer un gran lanzamiento. El mínimo viable en este caso sería tomar una muestra de tus clientes y ofrecerles el nuevo producto para estudiar su aceptación (o no).
    \item \textbf{Paso \#4: Analiza tus métricas}. Digamos que para probar nuestra hipótesis, vamos a un evento de navidad y ponemos un stand donde vendemos el producto que estamos probando. Para hacer nuestro estudio más robusto, podemos poner otro stand con nuestra oferta regular. De esta manera podremos comprobar la diferencia en las ventas.
    \item \textbf{Paso \#5: Toma acción}. Si las ventas en el stand con el producto nuevo son más altas que en el stand regular, significa que realmente hay interés en el producto novedoso.
\end{itemize}
Los libros tienen un problema: estas líneas las estás leyendo de forma lineal.

Da la impresión de que lo que te estoy diciendo es lineal, pero en realidad es cíclico. Vas a tomar acción y formular hipótesis de forma contínua. Si haces esto sin parar, estás reduciendo riesgos de negocio y mejorando tus métricas de negocio.

Te lo aseguro.

% --- INICIO DEL C-ODIGO PARA EL FINAL DEL CAP-ITULO ---

\begin{fullwidth}
\section*{Resumen del capítulo}

En este capítulo, cambiamos el enfoque de los modelos estadísticos a la filosofía que los pone en acción: una metodología para construir y mejorar un negocio iterando a partir de los datos.

Lo que vimos fue la idea central de \gls{lean-startup} y su mantra: ``fracasa rápido, fracasa barato''. Usamos la analogía de Thomas Bayes y las bolas de billar para entender el proceso: comenzamos con una creencia (nuestra idea de negocio) y usamos experimentos (lanzar un \gls{mvp}, medir, aprender) para actualizar continuamente esa creencia y acercarnos a la verdad de lo que el mercado realmente quiere. Aprendimos a distinguir entre las métricas que solo inflan el ego (\gls{vanity-metric}) y las que de verdad nos guían para tomar decisiones (accionables), como el \gls{cac}.

Esto es importante porque un plan de negocios no es más que una hipótesis esperando ser refutada. Las mejores herramientas de econometría son inútiles si no tienes un sistema para hacer las preguntas correctas y aprender de las respuestas de forma sistemática. Esta metodología te enseña que el fracaso no es el fin, sino una fuente valiosa de datos. Es la mentalidad que conecta la rigurosidad de la inferencia causal con la velocidad y agilidad que demandan los negocios modernos.

¿Cómo te ayuda esto? Este capítulo te da un mapa para aplicar todo lo que has aprendido. Tienes un ciclo claro —hipótesis, MVP, medir, actuar, repetir— que puedes usar para cualquier proyecto, ya sea lanzar una empresa desde cero, un nuevo producto, o simplemente mejorar una campaña de marketing. Te enseña a no enamorarte de tu plan, sino del proceso de aprendizaje. Ahora puedes proponer no solo un análisis, sino un `experimento de bajo costo` para resolver una duda de negocio, y sabes qué tipo de métricas debes observar para que la decisión sea la correcta.

\section*{Iterando hacia el éxito: ejercicios de estrategia}

La teoría es fácil, pero la verdadera lección de este capítulo está en la acción. Estos ejercicios son para que pienses como un estratega que usa los datos como su principal consejero.

\begin{enumerate}
    \item \textbf{De la Gran Idea al MVP (Conceptual):} Tu sueño es abrir una cafetería de especialidad con un ambiente único, granos de café exóticos y un espacio de coworking. Es un proyecto grande y caro. Describe un \gls{mvp} para esta idea. ¿Cuál es la versión más simple y barata que podrías lanzar en un mes para probar la hipótesis más importante: ``la gente de mi barrio está dispuesta a pagar más por un café de alta calidad''?

    \item \textbf{Vanidad vs. Acción (Conceptual):} Para tu proyecto de cafetería, clasifica las siguientes métricas como de Vanidad o Accionables y justifica por qué:
    \begin{itemize}
        \item a) El número de seguidores en la cuenta de Instagram de la cafetería.
        \item b) El número de clientes que regresan por segunda vez en una semana.
        \item c) El costo de los anuncios en redes sociales dividido entre el número de clientes nuevos que esos anuncios generaron (\gls{cac}).
        \item d) El número de `likes` en la foto de un latte art.
    \end{itemize}

    \item \textbf{El Ciclo de Iteración (Conceptual):} Tu MVP de la cafetería (ejercicio 1) fue un carrito de café por las mañanas. Los datos muestran que vendes mucho café, pero casi nadie compra tus granos exóticos de \$30 la bolsa; la mayoría pide el café de la casa. Ahora estás en el paso de ``Tomar acción''. ¿Cuál sería la siguiente `iteración` o `experimento` lógico para seguir aprendiendo sobre tu mercado?

    \item \textbf{La Analogía Bayesiana (Conceptual):} Explica la metodología de `Lean Startup` usando la analogía de las bolas de billar de Thomas Bayes.
    \begin{itemize}
        \item ¿Qué representa la ``bola blanca'' cuya posición es desconocida?
        \item ¿Qué representan las ``bolas rojas'' que lanzas a la mesa?
        \item ¿Qué es la ``información'' que obtienes de cada bola roja?
        \item ¿Cómo ``actualizas tu creencia'' sobre la posición de la bola blanca con esa información?
    \end{itemize}

    \item \textbf{Diseñando un Experimento de Producto (Conceptual):} Tienes una panadería y quieres probar si una nueva línea de pan de masa madre sin gluten tendría éxito. En lugar de reformar toda tu panadería (el plan a largo plazo), diseña un \gls{mvp}. ¿Cómo podrías probar esta hipótesis en un solo fin de semana y con una inversión mínima? ¿Qué métrica accionable medirías para decidir si la idea tiene futuro?

    \item \textbf{``No talk, all action'' (Reto Personal):} Piensa en un pequeño proyecto o idea que hayas tenido y dejado en el tintero. Describe el MVP más simple que podrías lanzar en una semana para dejar de planear y empezar a actuar y obtener datos reales.

    \item \textbf{Pivotar o Perseverar (Conceptual):} Tu MVP del pan sin gluten (ejercicio 5) revela algo curioso: casi nadie compró el pan, pero docenas de personas te preguntaron si podías venderles la masa madre activa para hacer pan en casa. En la jerga de `Lean Startup`, esta es una señal para ``pivotar''. Explica qué significa y cuál sería la nueva hipótesis principal de tu negocio.

    \item \textbf{Midiendo el CAC (Cálculo Simple):} Para promover tu nuevo servicio de masa madre, inviertes \$100 en un anuncio en un blog de cocina. Ese anuncio genera 40 clics a tu página. De esos 40 visitantes, 10 te dejan su correo electrónico. De esos 10 correos, 2 te hacen una compra de \$15 cada uno. ¿Cuál es tu Costo de Adquisición de Cliente (\gls{cac})? ¿Fue rentable esta campaña?

    \item \textbf{Los Mercados Eficientes y la Experimentación (Conceptual):} El capítulo menciona la \gls{hipotesis-mercados-eficientes}. Explica con tus propias palabras por qué esta idea económica refuerza la necesidad de experimentar en lugar de solo buscar una ``fórmula secreta'' para un negocio.

    \item \textbf{El Fracaso es Información (Conceptual):} La cita de Ed Catmull dice que el fracaso ``es una consecuencia necesaria de hacer algo nuevo''. ¿Cómo se conecta esta idea con el proceso de iteración? ¿Por qué un experimento que ``falla'' (es decir, que prueba que tu hipótesis era incorrecta) es increíblemente valioso?
\end{enumerate}
\end{fullwidth}

% --- FIN DEL CÓDIGO PARA EL FINAL DEL CAPÍTULO ---
