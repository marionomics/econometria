\chapter{Efectos Fijos}

\begin{quote}
\textit{I've learned a lot. \\
 And one of the things I've learned is life is really unpredictable. \\ And people can make forecasts, and they can make predictions. \\
 But those predictions and forecasts may not come true \\ if there's an unforseeable factor involved}\\
-- Taylor Swift en el iHeartRadio Music Awards en el día de $\pi$ de 2019.
\end{quote}

La regresión lineal es un modelo muy importante porque permite establecer controles en nuestros datos. El problema es que depende de un supuesto clave: inconfundibilidad condicional\sidenote{Las variables del lado izquierdo son los contrafactuales, la $T$ es el tratamiento y la $X$ son las covariables que usamos como control.}:
$$(Y_0,Y_1)\perp T|X$$
En otras palabras, requiere que todas las variables de confusión sean conocidas y medidas de tal manera que podamos incluirlas en el modelo y hacer que el grupo de tratamiento se comporte como si hubiera sido fruto de una asignación aleatoria \sidenote{En otras palabras, para cada observación, el resultado observado se puede expresar como $Y = T\cdot Y^1 + (1-T)\cdot Y^0$. Esto implica que la $Y$ y la $T$ son dependientes porque el valor promedio de $T\times Y^1$ no equivale al promedio de $(1-T)\cdot Y^0$}. Pero, a pesar de que no siempre tenemos el lujo de que nuestras variables sean observables, siempre podemos agruparlos con características en común.

Ese es el problema que resuelven los modelos de \gls{datos-panel}.

\section{Cómo se ven los datos en panel}
Imagina que estamos estudiando el efecto que hay entre el gasto en publicidad y los ingresos que nos genera.

Para ser mas claros, estamos haciendo una campaña para incrementar las ventas de una e-commerce por tres canales de venta. La primera es por anuncios de Google, el segundo con anuncios en Meta (que incluye Facebook e Instagram) y el tercero es por mail marketing.

Comencemos cargando nuestra base de datos en panel.

\begin{tcolorbox}[colback=gray!10, colframe=gray!10, breakable]
    \begin{minted}[frame=leftline, framesep=2mm, fontsize=\small]{python}
import pandas as pd
import statsmodels.formula.api as sm
    
df = pd.read_csv("sales-panel.csv")
df.head()
    \end{minted}
\end{tcolorbox}

Esto es el encabezado de los datos que nos genera el código de arriba.

\begin{table}[h]
    \centering
    \begin{tabular}{ccccc}
        \toprule
        & Medio       & Año  & Costo de publicidad & Ventas \\
        \midrule
        0 & Google Ads & 2020 & 1.25    & 3.4   \\
        1 & Google Ads & 2021 & 2.00    & 10.0  \\
        2 & Google Ads & 2022 & 6.00    & 13.5  \\
        3 & Google Ads & 2023 & 5.00    & 8.0   \\
        4 & Google Ads & 2024 & 6.00    & 11.0  \\
        \bottomrule
    \end{tabular}
    \caption{Datos de ventas y costos de anuncios en Google Ads}
    \label{tab:ads_sales}
\end{table}

Algunas observaciones sobre los datos:
\begin{itemize}
    \item Estamos agrupando los datos por el medio en el que se hace la publicidad (Anuncios de Google, Anuncios de Facebook y una campaña de email marketing). Eso quiere decir que la base de datos va a repetir cada uno de esos medios en la primera columna\sidenote{Esta forma de presentar los datos es menos visual que si extendiéramos cada uno de los medios en una columna cada uno (\emph{wide format}), pero hace más fácil aplicar los modelos que veremos en adelante.}.
    \item El supuesto clave en estos datos es que los clientes que obtenemos a partir de medios diferentes son distintos entre sí\sidenote{Este supuesto tiene sentido, simplemente porque se interactúa diferente en diferentes medios. El medio entonces agrupa muchas características no observables de los grupos. Por ejemplo, podríamos imaginar que los clientes de mail marketing ya confían más en nuestro contenido, mientras que los que vienen de publicidad de Meta podrían desconfiar aún en nosotros. La confianza no se puede medir, pero queda implícita en el medio.}.
    \item Es un ejemplo simple, pero en realidad un análisis de panel como este podría ser muy útil para analizar campañas distintas que corren en paralelo.
\end{itemize}

Hagamos un diagrama de dispersión para analizar los datos.

Lo que deseamos conocer es el efecto que hay entre el costo de la publicidad y las ventas de esa campaña particular.

\begin{tcolorbox}[colback=gray!10, colframe=gray!10, breakable]
    \begin{minted}[frame=leftline, framesep=2mm, fontsize=\small]{python}
import pandas as pd
import matplotlib.pyplot as plt
import numpy as np
import seaborn as sns

# Cargar los datos (ajusta la ruta si es necesario)
df = pd.read_csv("sales-panel.csv")

# Asignar un marcador distinto a cada medio
markers = {
    'Google Ads': 'o',      # Círculo
    'Facebook Ads': 's',    # Cuadrado
    'Email Marketing': 'D'  # Rombo
}

# Estilo blanco con cuadrícula
sns.set(style="whitegrid")

# Crear el gráfico
plt.figure(figsize=(10, 6))

for medio in df['Medio'].unique():
    subset = df[df['Medio'] == medio]
    plt.scatter(
        subset['Ad cost'], 
        subset['Sales'], 
        s=100, 
        label=medio, 
        color='black', 
        marker=markers[medio]
    )

# Calcular y graficar la línea de regresión general
X = df['Ad cost'].values
Y = df['Sales'].values
model = np.polyfit(X, Y, 1)
predicted = np.polyval(model, X)
plt.plot(X, predicted, color='gray', linewidth=2)

# Personalización del gráfico
plt.title('Diagrama de dispersión del costo de publicidad vs ventas')
plt.xlabel('Costos de publicidad')
plt.ylabel('Ventas')
plt.legend()
plt.grid(True)
plt.show()
    \end{minted}
\end{tcolorbox}

\begin{figure}
    \caption{En una regresión lineal que no distingue los datos entre diferentes medios, no parece haber ningún efecto entre el gasto de publicidad y las ventas.}
    \includegraphics[width=\linewidth]{imagenes/publicidad.png}
\end{figure}


Le puse formas distintas para que notes a simple vista: una regresión simple no es lo que deseamos hacer.

Si hiciéramos una regresión lineal simple tendríamos que nuestro gasto en publicidad no está aumentando las ventas. Al contrario, ¡Las está haciendo caer! Pero al separarlos por medio nos podemos dar cuenta de que no es así: cada una de las campañas de manera individual tiene un efecto positivo claro en las ventas.

Esto se ve más claro si trazamos una línea de regresión para cada uno de los medios en nuestro diagrama de dispersión.

\begin{fullwidth}
\begin{tcolorbox}[colback=gray!10, colframe=gray!10, breakable]
\begin{minted}[frame=leftline, framesep=2mm, fontsize=\small]{python}
# Identificar los medios únicos para asegurarnos de asignar un marcador único para cada uno
unique_media = df['Medio'].unique()
# Crear un diagrama de dispersión con diferentes marcadores para cada medio y líneas de regresión separadas
sns.lmplot(
    x='Ad cost', 
    y='Sales', 
    data=df, 
    hue='Medio', 
    markers=['o', 's', 'D', '^'][:len(unique_media)],
    palette = ['black'] * len(df['Medio'].unique()), 
    height=6, aspect=1.6, ci=None)

# Configurar el título y las etiquetas del gráfico
plt.title('Diagrama de dispersión del costo de publicidad vs ventas con líneas de regresión separadas por medio')
plt.xlabel('Costos de publicidad')
plt.ylabel('Ventas')
plt.show()
\end{minted}
\end{tcolorbox}
\end{fullwidth}

\begin{fullwidth}
\begin{figure*}
    \centering
    \caption{Sólo es necesario incluír el medio como variable de control. Hacerlo delata la verdadera relación que hay entre la publicidad y las ventas.}
    \includegraphics[width=1\linewidth]{imagenes/lineas_separadas.png}
\end{figure*}
\end{fullwidth}

\section{Los \gls{efectos-fijos} son sólo una aglomeración de variables de control en una sola}
La clave de los efectos fijos es que:
\begin{itemize}
    \item Podemos incluir el efecto del tiempo en nuestro modelo, pero el tiempo en sí mismo no es una variable.
    \item Lo que importa es que estamos capturando todas las características intrínsecas de nuestro medio y estamos asumiendo que son “fijas”.
\end{itemize}

El modelo de efectos fijos se define en términos generales como
$$Y_{it}=\beta X_{it}+\gamma U_{i}+\epsilon_{it}$$
donde $Y_{it}$ es el resultado que tiene el individuo $i$ en el tiempo $t$, que puede medirse en meses, años, trimestres o lo que sea que tenga sentido. Nuevamente, $X_{it}$ es el vector de variables para el individuo $i$ en el tiempo $t$\sidenote{Es muy común aprender sobre datos en panel con zonas geográficas. Por ejemplo, con registros de diferentes países en el tiempo.}.

Nota que ahora incluimos una variable $U_i$ que no tiene subíndice de tiempo $t$.

Esta representa el conjunto de inobservables del individuo $i$\sidenote{la $U$ es porque en inglés se dice \textit{unobservables}.}. Este elemento no tiene subíndice $t$ porque asumimos que estos inobservables no tienen variación en el tiempo. Por ejemplo, en una campaña de anuncios de google podemos asumir que el algoritmo que subasta un término de búsqueda es el mismo para todas las observaciones que hacemos.

\section{Variación dentro del individuo (\textit{within})}
En teoría, los efectos fijos funcionan igual que si usáramos una variable dummy para cada uno de los individuos.

El problema es que no es raro que nuestro panel se componga de más de 3 variables como en el ejemplo. Imaginemos que estamos tratando de hacer un panel para una campaña gigantesca ultrasegmentada de contenido con facebook ads. Una campaña así funcionaría haciendo un anuncio para cada pieza de contenido que hacemos, pautando (haciendo publicidad) y dejando que el algoritmo de meta encuentre a los consumidores ideales de ese contenido. Lo que obtenemos es una campaña para cada pieza de contenido que corre en paralelo. Esto puede hacer que el tamaño de nuestra base de datos aumente muy rápido\sidenote{Si de verdad usáramos \textit{dummies} para cada uno de los individuos, serían $n-1$ variables adicionales las que tendríamos que incluír en nuestra regresión. Si estás usando R como software para hacer tu trabajo y si incluyes la variable de clasificación como un objeto de tipo \codebox{factor}, el software lo transforma en \textit{dummies} de forma automática y hace la regresión con la función \codebox{lm()}.}.

Por eso haremos un pequeño truco que nos permitirá obtener el mismo resultado que si usaramos variables \textit{dummy}, pero con un conjunto más manejable de datos.

El primer paso es obtener el valor de la media condicional de publicidad y ventas en cada uno de los medios. Nuestro objetivo será identificar el punto en el que se cruzan los puntos medios de cada grupo para posteriormente juntarlos en un mismo medio.

Así se ven de manera visual.

\begin{fullwidth}
\begin{tcolorbox}[colback=gray!10, colframe=gray!10, breakable]
\begin{minted}[frame=leftline, framesep=2mm, fontsize=\small]{python}
# Calculando el promedio y la desviación estándar del costo de anuncios y las ventas para cada medio
media_stats = df.groupby('Medio').agg({'Ad cost': ['mean', 'std'], 'Sales': ['mean', 'std']}).reset_index()

# Creando el gráfico
plt.figure(figsize=(10, 6))
sns.scatterplot(x='Ad cost', y='Sales', data=df, hue='Medio', style='Medio', markers=['o', 's', 'D'][:len(unique_media)])

# Añadiendo las líneas para representar el promedio ± una desviación estándar
for _, row in media_stats.iterrows():
    medio = row['Medio']
    ad_cost_mean = row[('Ad cost', 'mean')]
    ad_cost_std = row[('Ad cost', 'std')]
    sales_mean = row[('Sales', 'mean')]
    sales_std = row[('Sales', 'std')]
    
    # Dibujando líneas para el costo de anuncios
    plt.plot([ad_cost_mean - ad_cost_std, ad_cost_mean + ad_cost_std], [sales_mean, sales_mean], 
             color='black', linestyle='-', linewidth=1)
    
    # Dibujando líneas para las ventas
    plt.plot([ad_cost_mean, ad_cost_mean], [sales_mean - sales_std, sales_mean + sales_std], 
             color='black', linestyle='-', linewidth=1)

plt.title('Diagrama de Dispersión de Costo de Anuncio vs. Ventas con Promedios y Desviaciones Estándar')
plt.xlabel('Costo de Anuncio')
plt.ylabel('Ventas')
plt.legend(title='Medio')
plt.grid(True)
plt.show()   
\end{minted}
\end{tcolorbox}
\end{fullwidth}

\begin{fullwidth}
\begin{figure*}[h!]
    \centering
    \caption{Cada medio de publicidad tiene un valor promedio de ventas ($\bar Y_i$) y del costo del anuncio ($\bar X_i$). Las líneas en el gráfico muestra donde se cruzan esos valores promedio, con una desviación estándar para determinar el tamaño de la línea. Este es un paso determinante para crear nuestras variables \emph{within}.}
    \includegraphics[width=1\linewidth]{imagenes/averages.png}
\end{figure*}
\end{fullwidth}

El siguiente paso es restar estas medias de nuestros individuos.
\begin{align*}
\ddot{Y}_{it}&=Y_{it}-\bar{Y}_{i}\\
\ddot{X}_{it}&=X_{it}-\bar{X}_{i}
\end{align*}
Visualmente lo que esto logra es como si “empalmaramos” las cruces que se formaron en el gráfico anterior.

Ahora sólo tenemos que hacer una regresión de $\ddot{Y}_{it}$ contra $\ddot{X}_{it}$, o bien
\begin{align*}
    (Y_{it}-\bar{Y}_{i})&=\beta(X_{it}-\bar{X}_{i})+(\epsilon_{it}-\bar{\epsilon}_{i})\\
    \ddot{Y}_{it}&=\beta\ddot{X}_{it}+\ddot{\epsilon}_{it}
\end{align*}

Estas son las nuevas variables de nuestra regresión de efectos fijos.

Nota que el término de elementos inobservables desaparece. Esto es porque $U_{i}=\bar{U}_{i}$, por su propia definición. Es una operación que elimina todos los términos constantes en el tiempo.

En concreto, la regresión se hace sobre este conjunto de datos.

\begin{fullwidth}
\begin{tcolorbox}[colback=gray!10, colframe=gray!10, breakable]
\begin{minted}[frame=leftline, framesep=2mm, fontsize=\small]{python}
df['Within Ad cost'] = df.groupby('Medio')['Ad cost'].transform(lambda x: x - x.mean())
df['Sales Within'] =  df.groupby('Medio')['Sales'].transform(lambda x: x - x.mean())
df.head()
\end{minted}
\end{tcolorbox}
\end{fullwidth}


\begin{table}[ht]
\centering
\caption{Gasto en publicidad y ventas por año (Google Ads)}
\label{tab:google_ads}
\begin{tabular}{cccccc}
\toprule
Año & Gasto publicitario & Ventas & Gasto interno & Ventas internas \\
\midrule
2020 & 1.25 & 3.4  & -2.625  & -6.42 \\
2021 & 2.00 & 10.0 & -1.875  &  0.18 \\
2022 & 6.00 & 13.5 &  2.125  &  3.68 \\
2023 & 5.00 & 8.0  &  1.125  & -1.82 \\
2024 & 6.00 & 11.0 &  2.125  &  1.18 \\
\bottomrule
\end{tabular}

\vspace{1mm}
\begin{flushleft}
\footnotesize
\textit{Notas:} El ``Gasto interno'' y ``Ventas internas'' corresponden a las variables centradas dentro del medio (\gls{within}). Todos los valores corresponden a la plataforma Google Ads.
\end{flushleft}
\end{table}


Y visualmente, podemos observar que colocamos juntos los centros que calculamos con anterioridad. Nuestro modelo es una regresión lineal con los datos modificados de esta manera:

\begin{fullwidth}
\begin{tcolorbox}[colback=gray!10, colframe=gray!10, breakable]
\begin{minted}[frame=leftline, framesep=2mm, fontsize=\small]{python}
# Calculando las estadísticas necesarias para los cruces en el gráfico
media_within_stats = df.groupby('Medio').agg({'Within Ad cost': ['mean', 'std'], 'Sales Within': ['mean', 'std']}).reset_index()

# Creando el gráfico con las variables "Within"
plt.figure(figsize=(10, 6))

# Usando scatterplot para los puntos con color, pero sin añadir la línea de regresión aquí
sns.scatterplot(x='Within Ad cost', y='Sales Within', data=df, hue='Medio', style='Medio', markers=['o', 's', 'D'][:len(df['Medio'].unique())])

# Añadiendo la línea de regresión con regplot
sns.regplot(x='Within Ad cost', y='Sales Within', data=df, scatter=False, color='gray')

# Añadiendo las cruces que representan la media ± una desviación estándar para las variables within
for _, row in media_within_stats.iterrows():
    medio = row['Medio']
    within_ad_cost_mean = row[('Within Ad cost', 'mean')]
    within_ad_cost_std = row[('Within Ad cost', 'std')]
    sales_within_mean = row[('Sales Within', 'mean')]
    sales_within_std = row[('Sales Within', 'std')]
    
    # Dibujando líneas para Within Ad cost
    plt.plot([within_ad_cost_mean - within_ad_cost_std, within_ad_cost_mean + within_ad_cost_std], [sales_within_mean, sales_within_mean], 
             color='black', linestyle='-', linewidth=1)
    
    # Dibujando líneas para Sales Within
    plt.plot([within_ad_cost_mean, within_ad_cost_mean], [sales_within_mean - sales_within_std, sales_within_mean + sales_within_std], 
             color='black', linestyle='-', linewidth=1)

plt.title('Diagrama de Dispersión de Within Ad Cost vs. Sales Within con Cruces y Única Línea de Regresión')
plt.xlabel('Within Ad Cost')
plt.ylabel('Sales Within')
plt.legend(title='Medio')
plt.grid(True)
plt.show()
\end{minted}
\end{tcolorbox}
\end{fullwidth}

\begin{fullwidth}
\begin{figure*}[h!]
    \centering
    \caption{Al absorber los efectos fijos, los datos se conjuntan en un solo punto medio de publicidad y ventas. El resultado es que el coeficiente de una regresión conjunta captura el efecto independiente del medio que se está usando.}
    \includegraphics[width=1\linewidth]{imagenes/within.png}
\end{figure*}
\end{fullwidth}

Nota cómo ahora las medias están en cero en los dos ejes.

Ahora todos los datos están en un punto comparable. A esto se le llama “absorber” los efectos fijos.

Naturalmente, ahora podemos aplicar una regresión lineal simple a nuestros datos.
\begin{tcolorbox}[colback=gray!10, colframe=gray!10, breakable]
\begin{minted}[frame=leftline, framesep=2mm, fontsize=\small]{python}
import statsmodels.api as sm
import statsmodels.formula.api as smf

# Define la fórmula para el modelo de efectos fijos con las variables centradas respecto a su grupo
formula = 'Q("Sales Within") ~ Q("Within Ad cost")'

# Ajusta el modelo de efectos fijos
model = smf.ols(formula, data=df).fit()

# Muestra el resumen de los resultados
model.summary()   
\end{minted}
\end{tcolorbox}

Y listo. En la siguiente tabla mostramos los resultados de la regresión.


\begin{table}[ht]
\centering
\caption{Efecto de la publicidad en las ventas hecha con un modelo de mínimos cuadrados con las variables centradas por grupos.}
\label{tab:ols_advertising}
\begin{tabular}{lccc}
\toprule
 & Coeficientes & Error estándar & IC 95\% \\
\midrule
Costo de publicidad & 0.828 & 0.214 & [0.374, 1.282] \\
Intercepto & 0.000 & 0.517 & [-1.095, 1.095] \\
\midrule
$R^2$ & \multicolumn{3}{l}{0.483} \\
$R^2$ ajustada & \multicolumn{3}{l}{0.451} \\
N (observaciones) & \multicolumn{3}{l}{18} \\
\bottomrule
\end{tabular}
\end{table}



%\begin{table}[h!]
    %\centering
    %\caption{OLS Regression Results}
    %\begin{tabular}{llll}
    %    \toprule
    %    \textbf{Dep. Variable:} & Q("Sales Within" & \textbf{R-squared:} & 0.483 \\
    %    \textbf{Model:} & OLS &  \textbf{Adj. R-squared:} & 0.451 \\
    %    \textbf{Method:} & Least Squares &  \textbf{F-statistic:} & 14.96 \\
    %    \textbf{Date:} & Wed, 29 Feb 2024 &  \textbf{Prob (F-statistic):} & 0.00136 \\
    %    \textbf{Time:} & 09:07:24 &  \textbf{Log-Likelihood:} & -38.610 \\
    %    \textbf{No. Observations:} & 18 &  \textbf{AIC:} & 81.22 \\
    %    \textbf{Df Residuals:} & 16 &  \textbf{BIC:} & 83.00 \\
    %    \textbf{Df Model:} & 1 \\
    %    \midrule
    %    \textbf{Covariance Type:} & nonrobust \\
    %    \bottomrule
    %\end{tabular}

    %\vspace{0.5cm}

    %\begin{tabular}{lccccc}
    %    \toprule
    %    & \textbf{coef} & \textbf{std err} & \textbf{t} & \textbf{P$>|$t$|$} & \textbf{[0.025, 0.975]} \\
    %    \midrule
    %    Intercept & 0 & 0.517 & 0 & 1.000 & [-1.095, 1.095] \\
    %    Q("Within Ad cost") & 0.8283 & 0.214 & 3.868 & 0.001 & [0.374, 1.282] \\
    %    \bottomrule
    %\end{tabular}

    %\vspace{0.5cm}

    %\begin{tabular}{llll}
    %    \toprule
    %    \textbf{Omnibus:} & 0.334 &  \textbf{Durbin-Watson:} & 2.169 \\
    %    \textbf{Prob(Omnibus):} & 0.846 &  \textbf{Jarque-Bera (JB):} & 0.487 \\
    %    \textbf{Skew:} & -0.177 &  \textbf{Prob(JB):} & 0.784 \\
    %    \textbf{Kurtosis:} & 2.275 &  \textbf{Cond. No.:} & 2.41 \\
     %   \bottomrule
    %\end{tabular}
%\end{table}


La regresión sobre nuestros datos centrados es una regresión lineal simple. Podemos observar que la relación entre la publicidad y las ventas es positiva al ver el coeficiente. También es una relación significativa, de acuerdo al estadístico t y al \textit{p-value}.

Este es un modelo sencillo con sólo dos variables. Por eso podemos aplicar el truco de centrar las variables en grupos directamente y usar una regresión sencilla de mínimos cuadrados. Sin embargo, en modelos más complejos, tendrás que usar paquetería especializada para el manejo de datos en panel.

Este es el código para hacer la regresión de efectos fijos con la función \codebox{PanelOLS}, del módulo \codebox{linearmodels}.

\begin{fullwidth}
\begin{tcolorbox}[colback=gray!10, colframe=gray!10, breakable]
\begin{minted}[frame=leftline, framesep=2mm, fontsize=\small]{python}
from linearmodels.panel import PanelOLS

# Preparing the data: Setting 'Medio' and 'Año' as index
df_panel = df.set_index(['Medio', 'Año'])

# Specifying and fitting the model with entity effects (fixed effects)
panel_model = PanelOLS.from_formula('Sales ~ Q("Ad cost") + EntityEffects', data=df_panel)

# Fitting the model
panel_results = panel_model.fit()

# Displaying the results
panel_results.summary   
\end{minted}
\end{tcolorbox}
\end{fullwidth}

La siguiente tabla condensa la información que nos genera el reporte de nuestra regresión por panel.

\begin{table}[ht]
\centering
\caption{Modelo de efectos fijos (PanelOLS) – Variable dependiente: Ventas}
\label{tab:panel_ols}
\begin{tabular}{lccc}
\toprule
 & Coeficiente & Error estándar & IC 95\% \\
\midrule
Costo publicitario ($x_{it}$) & 0.828*** & 0.229 & [0.337, 1.319] \\
\midrule
$R^2$ (dentro)         & \multicolumn{3}{l}{0.483} \\
$R^2$ (entre)          & \multicolumn{3}{l}{0.667} \\
$R^2$ (global)         & \multicolumn{3}{l}{0.643} \\
N (observaciones)      & \multicolumn{3}{l}{18} \\
Número de entidades    & \multicolumn{3}{l}{3} \\
Número de periodos     & \multicolumn{3}{l}{6} \\
Efectos incluidos      & \multicolumn{3}{l}{Entidad (efectos fijos)} \\
Estadístico F (modelo) & \multicolumn{3}{l}{13.09 (p = 0.0028)} \\
\gls{poolabilidad} & \multicolumn{3}{l}{F(2,14) = 13.45, p = 0.0006} \\
\bottomrule
\end{tabular}

\vspace{1mm}
\begin{flushleft}
\footnotesize
\textit{Notas:} Estimaciones obtenidas mediante mínimos cuadrados para datos de panel con efectos fijos.\\
\textit{Niveles de significancia:} * $p<0.1$, ** $p<0.05$, *** $p<0.01$
\end{flushleft}
\end{table}

¿Qué significa todo esto? Aquí algunos puntos a los que poner atención:

\begin{itemize}

    \item El modelo muestra un valor de $R^2$ \textit{within} (dentro), de 0.483. Lo podemos interpretar como que un 48\% de la variación de las ventas se explica por el gasto en publicidad.
    \item La prueba de agrupabilidad rechaza la hipótesis nula de que se pueda agrupar a todos los medios en un intercepto común. Con esto validamos el uso de un modelo de efectos fijos por entidad, a diferencia de un modelo agrupado (que sería la regresión simple de todos los datos).
    \item El modelo estima un resultado positivo y significativo. Un aumento en una unidad del gasto en publicidad se asocia con un aumento de 0.828 unidades en ventas. Este resultado ya controla diferencias constantes entre medios.

\end{itemize}

Nada mal, ¿cierto?.

\section{El efecto fijo de dos vías}

Digamos que queremos hacer lo mismo no sólo para los individuos, sino también para el tiempo.

El resultado sería un modelo como este:
$$Y_{it}=U_{i}+U_{t}+\beta X_{it}+\epsilon_{it}$$
Lo que este modelo nos da es una estimación que permite comparar la variación entre individuos al mismo tiempo que \textit{entre años}.

Por ejemplo, nota que en los datos anteriores, las ventas del año 2020 son relativamente más bajas que las demás. Es un recordatorio de la época de pandemia, en la que cerraron todos los negocios y las ventas de muchas cosas bajaron, a menos de que vendas cubrebocas. En el caso de las ventas por Facebook, el nivel de ventas que muestran los datos es muy bajo para lo que estamos acostumbrados a vender por ese medio, pero no es un nivel de ventas tan fuera de lo normal para el marketing por e-mail.

Usa este código para hacer una regresión de panel por dos vías.
\begin{tcolorbox}[colback=gray!10, colframe=gray!10, breakable]
\begin{minted}[frame=leftline, framesep=2mm, fontsize=\small]{python}
from linearmodels.panel import PanelOLS

# Efectos fijos de dos vias
model = PanelOLS.from_formula('Sales ~ Q("Ad cost") + EntityEffects + TimeEffects', data=df_panel)
results = model.fit()
print(results.summary)
\end{minted}
\end{tcolorbox}

Nuevamente, veamos un resumen de los resultados en el cuadro \ref{tab:panel_ols_sin_efecto}.

\begin{table}[ht]
\centering
\caption{Modelo de efectos fijos (PanelOLS) – Variable dependiente: Ventas}
\label{tab:panel_ols_sin_efecto}
\begin{tabular}{lccc}
\toprule
 & Coeficiente & Error estándar & IC 95\% \\
\midrule
Costo publicitario ($x_{it}$) & -0.089 & 0.322 & [-0.817, 0.639] \\
\midrule
$R^2$ (dentro)         & \multicolumn{3}{l}{-0.109} \\
$R^2$ (entre)          & \multicolumn{3}{l}{-0.142} \\
$R^2$ (global)         & \multicolumn{3}{l}{-0.138} \\
N (observaciones)      & \multicolumn{3}{l}{18} \\
Número de entidades    & \multicolumn{3}{l}{3} \\
Número de periodos     & \multicolumn{3}{l}{6} \\
Efectos incluidos      & \multicolumn{3}{l}{Entidad y tiempo (efectos fijos)} \\
Estadístico F (modelo) & \multicolumn{3}{l}{0.08 (p = 0.789)} \\
Prueba de agrupabilidad & \multicolumn{3}{l}{F(7,9) = 14.19, p = 0.0003} \\
\bottomrule
\end{tabular}

\vspace{1mm}
\begin{flushleft}
\footnotesize
\textit{Notas:} Estimaciones obtenidas mediante mínimos cuadrados para datos de panel con efectos fijos por entidad y por periodo.\\
\textit{Niveles de significancia:} * $p<0.1$, ** $p<0.05$, *** $p<0.01$
\end{flushleft}
\end{table}

Nota que Python permite hacer este tipo de modelos de una manera muy sencilla. Sólo es necesario incluír en la regresión los \codebox{EntityEffects} al mismo tiempo que los \codebox{TimeEffects}.


En esta tabla los resultados no parecen ser tan alentadores. El gasto en publicidad no tiene un efecto significativo con las ventas cuando controlamos por medio y por el tiempo a la vez.

Los \gls{efectos-fijos-dos-vias} son un tema muy interesante, pues capturan las características completas del año, a la vez que segmentan los efectos de los grupos en los que están clasificados los datos. Por eso este tipo de modelos son la base del modelo de \textbf{diferencias en diferencias}, que permiten diferenciar el efecto en el tiempo que tienen los grupos de ``tratamiento'' contra los de ``control''.

Lo que en la práctica quiere decir es que los modelos de diferencias en diferencias son modelos de efectos fijos de dos vías, pero no todos los efectos fijos de dos vías pueden generar un diseño de diferencias en diferencias.

¿Cuál es la diferencia? La forma en la que seleccionamos los grupos para que sean de tratamiento y de control.

En un modelo de diferencias en diferencias, lo que queremos es asemejar el diseño del estudio a un experimento natural.

Eso lo veremos en el siguiente capítulo.

% --- INICIO DEL CÓDIGO PARA EL FINAL DEL CAPÍTULO ---

\begin{fullwidth}
\section*{Resumen del capítulo}

En este capítulo nos enfrentamos a uno de los mayores enemigos de la inferencia causal: las variables importantes que no podemos ver ni medir.

Lo que hicimos fue aprender a usar una nueva estructura de datos, los \gls{datos-panel}, que nos permiten seguir a los mismos individuos (empresas, países, o en nuestro caso, medios de publicidad) a lo largo del tiempo. Vimos cómo una regresión simple puede mentirnos descaradamente, mostrándonos una relación negativa donde en realidad había una positiva. La solución fue aplicar la magia de los \gls{efectos-fijos}, un método que controla por todas esas características ``inobservables'' y constantes de cada individuo, como la ``calidad'' o la ``confianza'' inherente a cada medio de publicidad. Lo hicimos de dos formas: a mano, con la transformación \textit{within}, y luego con las herramientas profesionales de Python. Finalmente, subimos el nivel con los \gls{efectos-fijos-dos-vias} para controlar no solo por las diferencias entre individuos, sino también por los shocks que afectan a todos por igual en un año determinado.

Esto es importante porque la mayoría de los problemas del mundo real están plagados de heterogeneidad inobservable. La ``cultura'' de una empresa, la ``habilidad'' de un vendedor, o la ``calidad'' de una escuela son factores decisivos que casi nunca tenemos en una columna de nuestros datos. Ignorarlos lleva a conclusiones erróneas. El modelo de efectos fijos es tu primera gran herramienta para aislar un efecto causal en datos que no vienen de un experimento, permitiéndote hacer comparaciones mucho más justas.

¿Cómo te ayuda esto? Ahora tienes un método para analizar datos de individuos a lo largo del tiempo. Cuando sospeches que hay diferencias fundamentales y persistentes entre los grupos que comparas (ej. distintas sucursales de una tienda, distintos países, distintos productos), tu primer instinto será usar un modelo de efectos fijos. Te permite responder a la pregunta: ``Dentro de cada sucursal, ¿qué efecto tuvo cambiar X en el resultado Y, una vez que eliminamos las diferencias preexistentes entre la sucursal buena y la mala?''. Es una técnica increíblemente poderosa y es la base para entender el modelo de Diferencias en Diferencias que veremos a continuación.

\section*{Controlando el caos: ejercicios de panel y efectos fijos}

Es hora de aplicar esta poderosa técnica. Estos ejercicios te ayudarán a solidificar la intuición detrás de los efectos fijos y a practicar su implementación.

\begin{enumerate}
    \item \textbf{La trampa de la regresión simple (Conceptual):} El capítulo mostró cómo una regresión simple (o ``agrupada'') daba un resultado engañoso. Describe otro escenario de negocios donde podría pasar lo mismo. Por ejemplo, si analizaras la relación entre las horas trabajadas por los vendedores y las ventas totales, ¿qué ``efecto fijo'' a nivel de vendedor podría confundir los resultados si lo ignoras?

    \item \textbf{La magia de la transformación \textit{within} (Conceptual):} El truco de restar la media de cada individuo (``de-meaning'') elimina la variable inobservable $U_i$. ¿Por qué funciona esto? ¿Qué característica fundamental debe tener esa variable inobservable para que el truco de la resta la haga desaparecer?

    \item \textbf{Calculando el ``Within'' a mano (Código):} Usando el DataFrame \codebox{sales-panel.csv}, calcula manualmente las variables ``within'' para el medio \codebox{Facebook Ads}. Es decir, calcula la media de \codebox{Ad cost} y \codebox{Sales} \textit{solo para Facebook Ads}, y luego resta esas medias de los valores originales para obtener \codebox{Within Ad cost} y \codebox{Sales Within} para ese medio.

    \item \textbf{Requisitos para un modelo de panel (Conceptual):} Un colega te entrega un archivo de Excel y te pregunta si puede usar un modelo de efectos fijos. ¿Qué dos características clave debe tener esa tabla de datos para que tu respuesta sea ``sí''?

    \item \textbf{Efectos fijos por entidad (Código):} Carga el dataset \codebox{sales-panel.csv}. Usando la librería \codebox{linearmodels}, corre un modelo de efectos fijos por entidad (\codebox{EntityEffects}) para predecir \codebox{Sales} a partir de \codebox{Ad cost}. Muestra la tabla de resultados. Confirma que el coeficiente que obtienes para \codebox{Ad cost} es el mismo (0.828) que el que se obtuvo con la regresión sobre los datos transformados manualmente en el capítulo.

    \item \textbf{Efectos fijos por tiempo (Código y Conceptual):} Ahora, usando \codebox{linearmodels}, corre un modelo de efectos fijos \textit{solamente por tiempo} (\codebox{TimeEffects}), sin efectos de entidad. ¿Qué coeficiente obtienes para \codebox{Ad cost}? ¿Es significativo? ¿Qué te dice este resultado sobre la importancia de controlar por las características de cada \textit{medio} en lugar de controlar solo por el \textit{año}?

    \item \textbf{Interpretando el R-cuadrado ``Within'' (Conceptual):} El reporte de \codebox{PanelOLS} con efectos de entidad te da un $R^2$ \textit{within} de 0.483. Explica en una frase qué significa este número. ¿A qué porcentaje de la varianza se refiere?

    \item \textbf{El poder de los dos efectos (Código y Conceptual):} Corre el modelo de efectos fijos de dos vías (\codebox{EntityEffects + TimeEffects}) como se muestra en el capítulo. El resultado para \codebox{Ad cost} ahora no es significativo (p-value = 0.789). Propón una historia de negocio que explique por qué podría pasar esto. ¿Qué pudo haber ocurrido en ciertos años que, al controlarlo, ``absorbió'' el efecto que antes veíamos en la publicidad?

    \item \textbf{La prueba de `Poolability` (Conceptual):} En el resultado del modelo con solo efectos de entidad, la prueba de ``Poolability'' tiene un p-value muy bajo (0.0006). La hipótesis nula de esta prueba es que los efectos fijos de todos los medios son iguales entre sí. Dado el p-value, ¿rechazas o no rechazas la hipótesis nula? ¿Qué te dice esto sobre si fue buena idea usar efectos fijos en lugar de una regresión simple?

    \item \textbf{Reto - Estructurando datos de panel (Conceptual y Código):} Imagina que tienes un archivo CSV con esta estructura (formato ``ancho'' o \textit{wide}):
    
    \begin{tabular}{lcccc}
    \toprule
    Tienda & Ventas\_2022 & GastoPubli\_2022 & Ventas\_2023 & GastoPubli\_2023 \\
    \midrule
    A & 100 & 10 & 120 & 12 \\
    B & 200 & 15 & 210 & 16 \\
    \bottomrule
    \end{tabular}
    
    Los modelos de panel necesitan un formato ``largo'' (\textit{long}). Describe cómo se vería esta tabla en formato largo. (Pista: investiga la función \codebox{pd.melt} de pandas).
\end{enumerate}
\end{fullwidth}

% --- FIN DEL CÓDIGO PARA EL FINAL DEL CAPÍTULO ---