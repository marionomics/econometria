\chapter{Cómo hacer Investigación de mercados con inteligencia artificial}

\begin{quote}
\textit{What you call love was invented by guys like me to sell nylons}

-- Don Draper
\end{quote}

Todos los negocios necesitan hacer investigación de mercado.

Imagínate que te subes a un avión y el piloto te dice que los monitores de vuelo están descompuestos. No hay comunicación con las torres de control, no hay forma de saber cómo será el clima en el trayecto y no hay forma de saber si vas en la ruta correcta y llegarás al aeropuerto correcto (o a algún aeropuerto siquiera). ¿Te subes?

¡Por supuesto que no!

La investigación de mercado es la forma en la que un negocio obtiene información sobre su entorno. Puede ser de manera formal (con encuestas, entrevistas, \gls{focus-group} y observación de las métricas) o informal (la observación simple). Todas las empresas la tienen, porque todas reciben información de su entorno.

Son nuestros monitores de vuelo.

\section{Cuando una marca de pasteles instantáneos decidió quitarle ingredientes a su producto y ¡vendieron más que nunca}

\begin{quote}
\textit{A veces no hacer nada es hacerlo todo} \\
- Teniente Harina
\end{quote}


En los años 50s toda la economía en Estados Unidos estaba creciendo.

Cuando la economía está creciendo, tener un producto medianamente innovador era garantía de que te harías millonario. A pesar de eso, General Mills no lograba levantar las ventas de su mezcla para hacer pastel en casa. Era sin duda un producto innovador

Lo que hicieron a continuación, convirtió a las mezclas en un éxito y revolucionó el mercado para siempre.

La necesidad era real: las amas de casa estaban hambrientas de soluciones que les ahorren tiempo de las tareas del hogar. Así que los ejecutivos decidieron contratar a Ernest Dichter para hacer una \gls{investigacion-mercados}. Dichter fue el pionero de los \textbf{grupos de enfoque}, que juntan en un sólo cuarto a un grupo de amas de casa para encontrar las verdaderas causas del problema.

Tras entrevistar a más de cien amas de casa, Dichter encontró que la razón por la que no compraban era \textbf{culpa}. ¡Ellas querían participar en la elaboración del pastel! Un producto “demasiado fácil” de preparar las hacía ver cómo flojas y no cómo amas de casa de verdad (recuerda que eran los años 50).

La solución: quitar ingredientes a la mezcla y dejar que las amas de casa aporten más al proceso.

Ahora las amas de casa debían incluir huevo y leche a la mezcla para que quedara completa. Para la empresa esto significaba menos costos, pero más importante es que las ventas por fin crecieron. La mezcla era ahora más valiosa, porque permitía hacer un pastel, ser parte del proceso y terminarlo con facilidad.

Así de importante es tener un buen conocimiento del mercado.

\section{Pero ¿Un estudio de mercado es sólo hacer preguntas?}
Si, pero no es fácil hacer las preguntas correctas.

Para hacer las preguntas correctas necesitas entender el problema que estás solucionando, al cliente para quien estás haciendo tu solución y cómo estás contribuyendo a la solución. El problema y tu solución son importantes porque ayudan a identificar lo que debes cambiar para hacer tu oferta más valiosa, pero esas son cosas que están bajo tu control. Lo que no está control es tu cliente y sus decisiones.

Hay cosas de tu cliente que puedes aprender observando, pero otras se las vas a tener que preguntar directamente.
\begin{itemize}
    \item ¿Por qué te eligió en lugar de la competencia?
    \item ¿Cómo usa tu producto / servicio?
    \item ¿Percibe algún riesgo de que tu producto no cumpla su promesa?
\end{itemize}
En realidad el cielo es el límite.

Puedes preguntarle lo que se te ocurra a tus clientes. El problema es que la mayoría de nosotros no sabemos diseñar las preguntas correctas y los clientes no tienen ningún incentivo para decirnos la verdad. Hace un par de años, si no tenías el don de diseñar encuestas o no tenías a tu disposición a un buen sociólogo, mala suerte.

Pero hoy podemos apoyarnos de la IA para crear una encuesta increíble.

\section{Usemos a chatGPT para diseñar una encuesta en el estilo del Mom Test}
En 2013, Rob Fitzpatrick publicó “The Mom Test”.

En este libro, explicaba cómo cuando diseñamos una pregunta para una investigación de mercado, solemos mezclar objetivos y acabamos queriendo publicitar nuestro producto o servicio. El resultado es que hacemos las preguntas “amañadas” para que nos den una respuesta positiva. Pero eso no es útil para nosotros: queremos que si alguien odia nuestro producto, nos lo diga con todo lujo de detalle.

En otras palabras, quieres que incluso tu mamá te diga si hay algo en tu producto que necesitas cambiar.

El libro está lleno de ejemplos, geniales. Si los estudias, puedes transformar las preguntas en tu estudio de mercado para que cumplan con el objetivo y mejores la calidad de tus respuestas.

O puedes decirle a chatGPT que haga esa modificación por ti.

Estos son mis pasos:
\begin{itemize}
    \item Define lo que quieres conocer de tus clientes.
    \item Redacta o elige las preguntas que quieres hacerles.
    \item Transformalas al estilo del mom test.
\end{itemize}
Hagamos un ejemplo.

\subsection{Define lo que quieres conocer de tus clientes}
Digamos que tienes una e-commerce de ropa y quieres identificar puntos de fricción en el proceso de compra. En este caso hay miles de elementos que puedes probar por tu cuenta haciendo una \gls{abtesting}. Pero también hay aspectos que no puedes observar y que tienes que únicamente puedes saber si se los preguntas directamente al cliente.

Tres ejemplos de esto serían:
\begin{itemize}
    \item \textbf{La calidad percibida del producto}. ¿Por qué no la puedes medir con los datos de interacción? Como es una tienda en línea, hay detalles que las fotos no pueden mostrar. Algunos clientes lo podrían poner en las reseñas, pero muchos probablemente sólo dejen de comprar. Adios cliente y adiós datos.
    \item \textbf{Experiencia de soporte al cliente}. Esto incluye aspectos como la facilidad del menú, la rapidez y utilidad de las respuestas y la satisfacción general.
    \item \textbf{La experiencia del checkout}. Hay aspectos como qué tan fácil o difícil fue usar el carrito de compra, si les gustaría incluir opciones de pago o la claridad en costos de envío.
\end{itemize}
Puse ejemplos de cosas que no podrías medir usando pruebas A/B, pero tal vez también desees investigar aspectos que se pueden medir en pruebas A/B usando una encuesta.

La razón principal de esto es que el tiempo es limitado y sólo puedes hacer pruebas limitadas a la vez. Muchos expertos recomiendan sólo usar una prueba a la vez, pero los modelos de panel y de diferencias en diferencias en este libro te pueden ayudar a hacer más. Aún con esto, tal vez no quieras usar tu tiempo y recursos de las pruebas A/B en hacer estudios triviales.

\subsection{Redacta o elige las preguntas que quieres hacer}

Comencemos definiendo lo que queremos conocer.

Usa el siguiente prompt en chatGPT o el mejor modelo de inteligencia artificial al que tengas acceso.
\begin{tcolorbox}[colback=gray!10, colframe=gray!10, breakable]
\ttfamily
Ayúdame a hacer las preguntas para un estudio de mercado.

Giro: e-commerce de ropa.
Mercado meta: Gen-Z de Mexico y latinoamerica.
Propuesta unica de valor: Te damos puntos por tu ropa vieja que te dan descuentos en la nueva. Estamos luchando para combatir el fast fashion.

Te daré información sobre lo que quiero conocer de ellos y me darás 15 opciones de preguntas tocando los temas siguientes:

- La calidad percibida del producto.
- Experiencia de soporte al cliente.
Incluye aspectos como la facilidad del menú, la rapidez y utilidad de las respuestas y la satisfacción general.
- La experiencia del checkout. Hay aspectos como qué tan fácil o difícil fue usar el carrito de compra, si les gustaría incluir opciones de pago o la claridad en costos de envío. 
\end{tcolorbox}
Normalmente deseamos que la extensión de nuestro estudio sea suficientemente larga para que nos de información valiosa, pero no tan larga para que la contesten hasta acabarla (con sinceridad). Por eso le pedimos 15 preguntas, pero seleccionamos sólo las mejores. También toma en cuenta que la IA es muy buena para inventar, pero no todo lo que te de serán preguntas de calidad.

Modifica el prompt anterior con los datos de tu negocio o emprendimiento y sigamos.

\subsection{El mom test, explicado}
Te voy a ahorrar leer un libro entero en este texto de 3 minutos.

El mom test es una técnica donde las preguntas se diseñan para que no te contesten lo que piensan que quieres oír. Digamos que queremos preguntar cómo califican la calidad de la ropa en nuestra e-commerce. Hay muchas razones por las que alguien que conteste el cuestionario no quiere contestarte con la verdad\cite{momtest}.
\begin{itemize}
    \item Tal vez sienten que decir algo negativo sería muy confrontativo.
    \item Tal vez realmente quieren agradarte.
    \item O simplemente les estás haciendo una pregunta en la que tienen que pensar demasiado su respuesta y te acaban contestando lo que sea.
\end{itemize}
En todos estos casos, el culpable no es el que responde, eres tú.

Para entender mejor el mom test, observa estas preguntas. Son ejemplos de lo que no se debe hacer. ¿Puedes identificar por qué son malas preguntas?
\begin{itemize}
    \item ¿Crees que $\{X\}$ es buena idea de negocio?
    \item ¿Comprarías $\{X\}$?
    \item ¿A qué precio te parece justo comprar $\{X\}$?
\end{itemize}
Estas preguntas están mal planteadas porque alguien que no estaría dispuesto a comprar tu producto te puede fácilmente decir que sí, es buena idea. Tú te vas feliz con la impresión de que tienes la idea más genial de negocio y al final creas algo que a nadie le interesa. \textit{Sad to be you}.

En lugar de eso, puedes preguntar
\begin{itemize}
    \item ¿Cuáles son los mayores problemas que enfrentas cuando haces \{actividad relacionada con X\}?
    \item Cuéntame sobre la última vez que necesitaste algo similar a \{X\}. ¿Que solución utilizaste?
    \item ¿Qué has pagado anteriormente por productos o servicios similares a \{X\}? ¿Cómo decidiste que valía la pena ese precio.
\end{itemize}
La razón por la que estas preguntas son mejores es que están diseñadas para sacarle la verdad al encuestado.

Las preguntas del \textit{mom test} se enfocan en experiencias pasadas y decisiones reales. Hacer esto refleja mejor cómo se comportaría el usuario en una situación específica. Además, son preguntas que ayudan a entender el contexto y las motivaciones del usuario.

Ahora transformemos nuestras preguntas.

\section{Transformando nuestro estudio al estilo del mom test}
Para mi, el mom test es una metodología fácil de aprender, pero difícil de implementar.

Me toma mucho tiempo y trabajo hacer esas modificaciones. Afortunadamente los modelos de lenguaje grandes tienen ya cargada en su entrenamiento la información relacionada al mom test, porque hay miles de posts de blogs sobre el tema (y probablemente el libro mismo ahí venga cargado. Lo que antes nos tomaba horas, lo podemos hacer en minutos.

Puedes hacerle un prompt sencillo:
\begin{tcolorbox}[colback=gray!10, colframe=gray!10, breakable]
\begin{minted}[frame=leftline, framesep=2mm, fontsize=\small]{python}
redacta las preguntas en el estilo del mom test
\end{minted}
\end{tcolorbox}
Listo.

Tu encuesta ahora genera preguntas más interesantes. Son preguntas mejor diseñadas, que te generarán respuestas de mayor calidad. Puedes modificar más tu instrumento, pidiéndole aspectos específicos, como hacerlo en una \gls{escala-likert}\sidenote{Son de esas preguntas donde puntúas del 1 al 5 o del 1 al 7 qué tan de acuerdo estás con lo que se expresa.}.

\section{Cuándo no usar datos cuantitativos}

Hemos llegado al final de este libro.

Y como cierre, te quiero invitar a que \textbf{no uses lo que aprendiste aquí}. Al menos no si no tienes una buena razón de hacerlo.

En ocasiones, aprender este tipo de técnicas es emocionante y queremos aplicarlo cuanto antes y usarlo en todos los problemas que se nos presentan. Pero eso es como si yo te presentara un martillo y tú quisieras usarlo en todo tipo de problemas. Incluso en aquellos que claramente necesitan un desarmador.

Parece que te estoy diciendo algo obvio, pero lo he visto en más de una ocasión.

No todos los estudios de mercado necesitan diseñar una entrevista y aplicarla a un diseño muestral estadísticamente significativo. Algunas cosas simplemente requieren que salgas al mundo y le preguntes a las personas sobre tu producto. La técnica que te enseñé arriba te puede ayudar a diseñar preguntas de tipo \gls{mom-test}, pero no hay nada que sustituya salir al mundo real y preguntarle a las personas reales cómo se sienten.

La tecnología sigue avanzando, y las técnicas que podemos usar de inferencia causal seguirán ampliándose.

Con el avance de la IA, tendremos más implementaciones interesantes. Procesar la información de video, de audio, transcripciones y emociones abrirán la puerta a mucha investigación innovadora. Aprender a usar python y la IA nunca ha sido tan importante.

Ya diste el primer paso. Y por eso, te felicito.

% --- INICIO DEL CÓDIGO PARA EL FINAL DEL CAPÍTULO ---

\begin{fullwidth}
\section*{Resumen del capítulo}

En este capítulo final, cerramos el círculo. Vimos que, a pesar de todo el poder de los modelos que hemos aprendido, a veces la respuesta más profunda no está en los números, sino en una buena conversación.

Lo que hicimos fue explorar la investigación de mercados como el sistema de navegación de un negocio. La historia de la mezcla para pastel de General Mills nos enseñó una lección inolvidable: a veces, el problema no es el producto, sino una emoción humana (la culpa) que ninguna métrica cuantitativa puede detectar. Aprendimos la filosofía del \gls{mom-test} para diseñar preguntas que nos den la verdad cruda en lugar de cumplidos inútiles, enfocándonos en experiencias pasadas en vez de opiniones hipotéticas. Y lo más importante, descubrimos cómo usar la Inteligencia Artificial como un asistente experto para diseñar encuestas y entrevistas de alta calidad en minutos.

Esto es importante porque un negocio es, en esencia, una relación con sus clientes. Si solo miras los datos cuantitativos, solo estás escuchando la mitad de la historia. Entender las motivaciones, frustraciones y el contexto de tus clientes —el `insight` cualitativo— es lo que te permite crear valor verdadero. Este capítulo es el recordatorio crucial de que la inferencia causal no se trata solo de modelos, sino de una búsqueda honesta de la verdad, y a veces, el camino más corto es simplemente preguntar de la manera correcta.

¿Cómo te ayuda esto? Ahora tienes una herramienta para investigar el ``porqué'' detrás del ``qué'' que te muestran tus datos. Cuando tus métricas te señalen un problema (por ejemplo, ``la gente abandona el carrito de compra aquí''), tienes un método para descubrir la causa raíz hablando con tus clientes. Puedes usar la IA para generar rápidamente una guía de entrevista que te dará `insights` mucho más profundos y honestos. Esta es la pieza que conecta tu rigor analítico con la empatía humana, dándote una visión completa para tomar mejores decisiones y, al final, construir algo que a la gente realmente le importe.

\section*{La conversación final: ejercicios de investigación y cierre}

Has llegado al final. Estos últimos ejercicios son para solidificar el arte de hacer buenas preguntas y para reflexionar sobre el camino recorrido.

\begin{enumerate}
    \item \textbf{Identificando malas preguntas (Conceptual):} Un amigo está desarrollando una app para meditar y le pregunta a sus potenciales usuarios: ``¿Crees que una app que te ayude a reducir el estrés con meditaciones guiadas es una buena idea?''. Usando los principios del \gls{mom-test}, explica por qué esta pregunta, aunque bien intencionada, es inútil para validar el negocio.

    \item \textbf{Reformulando al estilo `Mom Test` (Conceptual):} Reescribe la pregunta del ejercicio anterior. Formula tres preguntas alternativas al estilo `Mom Test` que le darían a tu amigo información mucho más valiosa. (Pista: enfócate en el problema y en el comportamiento pasado. Por ejemplo: ``Cuéntame sobre la última vez que te sentiste estresado, ¿qué hiciste al respecto?'').

    \item \textbf{Tu socio IA (Práctico):} Elige un producto o servicio que uses con frecuencia. Imagina que tu objetivo es descubrir ``qué es lo que más frustra a los usuarios avanzados de este producto''. Sigue el flujo de trabajo del capítulo:
    \begin{itemize}
        \item Escribe un `prompt` para ChatGPT (o similar) pidiéndole 10 preguntas iniciales para una encuesta sobre este tema.
        \item Escribe un segundo `prompt` pidiéndole que transforme esas 10 preguntas al estilo del `Mom Test`.
        \item Comparte la mejor pregunta ``antes'' y su versión ``después'' del `Mom Test`.
    \end{itemize}

    \item \textbf{La lección del pastel (Conceptual):} La historia de la mezcla para pastel de General Mills es un caso clásico de investigación de mercados. ¿Cuál era el `insight` cualitativo clave que ninguna prueba A/B o análisis de ventas les estaba mostrando?

    \item \textbf{Cuantitativo y Cualitativo (Conceptual):} Estás analizando los datos de una plataforma de cursos en línea y observas un dato cuantitativo: ``El 80\% de los usuarios que compran un curso nunca lo terminan''. Ahora, formula tres preguntas cualitativas (estilo `Mom Test`) que le harías a esos usuarios para entender el ``porqué'' detrás de ese dato.

    \item \textbf{``No hacer nada es hacerlo todo'' (Conceptual):} El capítulo cita al Teniente Harina. ¿Cómo se aplica esta frase a la solución que encontró General Mills? ¿En qué sentido ``quitarle'' algo al producto fue la mejor decisión que pudieron tomar?

    \item \textbf{Saber cuándo guardar el martillo (Reflexión):} El capítulo termina con la advertencia de no usar estas técnicas para todo. Describe un problema de negocio donde correr un modelo de efectos fijos sería la herramienta `incorrecta` o excesiva, y donde una simple conversación con cinco clientes sería infinitamente más útil.

    \item \textbf{Tu propio `Mom Test` (Práctico/Reto):} Piensa en un proyecto real en el que estés trabajando (en tu empleo, estudios o un proyecto personal). Define una hipótesis clave que tengas sobre tus ``clientes'' o ``usuarios''. Escribe 3 preguntas al estilo `Mom Test` que podrías hacerle a alguien mañana mismo para empezar a validar esa hipótesis.

    \item \textbf{Conectando los puntos (Reflexión):} ¿Cómo se conecta la filosofía de `iteración` del capítulo anterior con la `investigación de mercados` de este capítulo? ¿En qué parte del ciclo ``hipótesis, MVP, medir, actuar'' pondrías las entrevistas a clientes?

    \item \textbf{Tu siguiente paso (Reflexión Personal):} Has llegado al final. ¡Felicidades! Este libro fue el primer paso. ¿Cuál es el concepto o la habilidad más importante que te llevas? Y más importante aún, ¿cuál será el `siguiente paso` que darás para seguir practicando y aprendiendo sobre inferencia causal, Python e IA?
\end{enumerate}
\end{fullwidth}

% --- FIN DEL CÓDIGO PARA EL FINAL DEL CAPÍTULO ---