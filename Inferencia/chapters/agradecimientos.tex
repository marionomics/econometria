\chapter{Agradecimientos}

\begin{quote}
\textit{No basé mi carrera en tener hits.\\
Tengo hits porque yo senté las base'}

- ROSALÍA
\end{quote}

No se puede escribir un libro si no es rodeado de una red de apoyo enorme.

Quiero dedicar esta página a darle su espacio a tantos a quienes agradezco y que han sido clave para la realización de esta obra.


En primer lugar están mi esposa María y mis hijos Román y Natalia, por su paciencia y por su apoyo. Al iniciar este proyecto, me pude robar el mejor espacio de la casa para hacerlo mi estudio y tuve la dicha que poder compartir momentos valiosos en familia en paralelo a la escritura de este texto. Masha en particular ha sido fuente de inspiración. Cada vez que por fin entiendo algo más sobre cómo funcionan y cómo hacer crecer los negocios, encuentro ejemplos de Masha ya aplicándolos.

Agradezco a los directivos de mi facultad. Desde el momento en el que comencé este proyecto a que se imprimió hubo un cambio de cuerpo directivo, pero en todo momento he sentido todo el respaldo y el apoyo. En su momento el Dr. José Ramón Duarte Carranza me dio sin ninguna duda ni miramientos el permiso para tomar mi sabático y embarcarme en este aventura, y ahora recibo con mucho aprecio el respaldo del Dr. Jesús Sotelo Asef para su publicación.

Tengo en la FECA uno de los trabajos más gratificantes que podría pedir, y agradezco de todo corazón a todos mis amigos y compañeros que me han acompañado en la Facultad. En particular a mis compañeros y amigos del cuerpo académico: César, Paco, Nacho, Julieta y Brenda, que me atrevo a mencionar ya sólo por nombre por la amistad que se ha forjado en los años de trabajo conjunto. Más allá de las paredes de la facultad, agradezco a la editorial UJED, con especial énfasis en Manuel: todos sabemos la labor titánica que haces por la editorial y por la facultad. Gracias también a los revisores anónimos, cuyas observaciones mejoraron sustancialmente este trabajo. Y un agradecimiento especial a Alina, que puso todo su corazón para que la portada quedara fantástica.

Pero uno de los más grandes agradecimientos es a mis alumnos de cuarto semestre de la carrera de Licenciado en Economía y Negocios Internacionales. Por años, los alumnos de ese grupo han sido mis conejillos de indias y me han permitido probar conceptos de una forma y otra. Este libro nació a partir de la clase de Aplicación de Principios Económicos, que desapareció del plan de estudios, pero que yo usé como una especie de pre-econometría.