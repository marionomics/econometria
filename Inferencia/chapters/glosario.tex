% ---------------- GLOSARIO CAUSAL ----------------
\newglossaryentry{inferencia-causal}{
  name={inferencia causal},
  sort=inferenciacausal,
  description={Disciplina que estudia cómo identificar y cuantificar relaciones de causa–efecto 
               a partir de datos observacionales o experimentales, combinando teoría estadística
               y supuestos sobre el proceso generador de datos}
}

\newglossaryentry{experimento-natural}{
  name={experimento natural},
  sort=experimentonatural,
  description={Situación del mundo real en la que eventos, políticas o cambios institucionales 
               generan una asignación \textit{casi} aleatoria de un tratamiento, lo que permite 
               estimar efectos causales sin un ensayo controlado tradicional}
}

\newglossaryentry{exceso-de-muertes}{
  name={exceso de muertes},
  sort=excesodemuertes,
  description={Diferencia entre el número observado de fallecimientos en un periodo y el número 
               esperado según patrones históricos; indicador para evaluar el efecto causal de 
               crisis sanitarias, desastres o políticas públicas}
}
% --------------------------------------------------

% --- GLOSARIO | Cap. “Los negocios son matemáticas” ------------------

\newglossaryentry{regresionlineal}{
  name={regresión lineal},
  sort=regresionlineal,
  description={Modelo estadístico que relaciona una variable dependiente
               con una o más variables independientes mediante una función
               lineal; punto de partida clásico para estimar efectos causales
               cuando se cumplen sus supuestos}
}

\newglossaryentry{diferenciasendiferencias}{
  name={diferencias en diferencias},
  sort=diferenciasendiferencias,
  description={Diseño cuasi-experimental que estima un efecto causal al
               comparar la evolución temporal de un grupo tratado y uno de
               control, bajo el supuesto de tendencias paralelas}
}

\newglossaryentry{grupodecontrol}{
  name={grupo de control},
  sort=grupodecontrol,
  description={Conjunto de unidades que no recibe el tratamiento en un
               experimento o cuasi-experimento; permite aproximar el resultado
               contrafactual y aislar el efecto causal}
}

\newglossaryentry{correlacioncausalidad}{
  name={correlación vs.\ causalidad},
  sort=correlacioncausalidad,
  description={Distinción fundamental entre asociaciones estadísticas
               (correlación) y relaciones de causa–efecto (causalidad); 
               confundirlas conduce a inferencias erróneas}
}

\newglossaryentry{pruebasdehipotesis}{
  name={pruebas de hipótesis},
  sort=pruebasdehipotesis,
  description={Procedimientos estadísticos que contrastan afirmaciones sobre
               parámetros mediante un estadístico y una regla de decisión; en
               inferencia causal se usan para evaluar la significancia de los
               efectos estimados}
}

\newglossaryentry{variablealeatoria}{
  name={variable aleatoria},
  sort=variablealeatoria,
  description={Función que asigna valores numéricos a los resultados de un
               experimento aleatorio; concepto base para describir
               distribuciones y estimadores en econometría causal}
}
% --------------------------------------------------------------------

% --- GLOSARIO | Cap. “Python para hacer Econometría” -----------------

\newglossaryentry{python}{
  name={Python},
  sort=python,
  description={Lenguaje de programación de propósito general, 
               ampliamente usado en ciencia de datos y econometría
               gracias a su sintaxis sencilla y a un ecosistema extenso
               de bibliotecas estadísticas}
}

\newglossaryentry{googlecolab}{
  name={Google Colab},
  sort=googlecolab,
  description={Servicio gratuito en la nube que ofrece cuadernos 
               Jupyter con GPU/CPU preconfiguradas y librerías
               de ciencia de datos; permite ejecutar código Python
               sin instalar nada en local}
}

\newglossaryentry{notebook}{
  name={notebook (cuaderno Jupyter)},
  sort=notebook,
  description={Documento interactivo que combina texto, código
               ejecutable y resultados; facilita el análisis
               econométrico iterativo y reproducible}
}

\newglossaryentry{pandas}{
  name={\texttt{pandas}},
  sort=pandas,
  description={Biblioteca de Python que provee estructuras de datos
               como \textit{Series} y \textit{DataFrame}; pilar para
               la limpieza, transformación y exploración de datos
               tabulares en econometría}
}

\newglossaryentry{dataframe}{
  name={DataFrame},
  sort=dataframe,
  description={Estructura tabular de \texttt{pandas} con índice y
               columnas etiquetadas; análoga a una hoja de cálculo
               pero optimizada para operaciones vectorizadas}
}

\newglossaryentry{numpy}{
  name={\texttt{NumPy}},
  sort=numpy,
  description={Paquete fundamental para computación numérica en
               Python; ofrece arrays multidimensionales y rutinas
               de álgebra lineal que subyacen a la mayoría de los
               modelos econométricos}
}

\newglossaryentry{statsmodels}{
  name={\texttt{statsmodels}},
  sort=statsmodels,
  description={Biblioteca de Python orientada a econometría y
               estadística; incluye regresiones lineales, modelos
               de series de tiempo y pruebas de hipótesis}
}

% --- Más entradas útiles --------------------------------------------

\newglossaryentry{stata}{
  name={Stata},
  sort=stata,
  description={Paquete estadístico propietario muy popular en econometría
               aplicada; destaca por su sintaxis compacta y amplia
               colección de comandos para modelos lineales, paneles
               y series de tiempo}
}

\newglossaryentry{eviews}{
  name={Eviews},
  sort=eviews,
  description={Software comercial orientado a análisis econométrico y
               pronóstico; integra interfaz gráfica para estimar modelos,
               ejecutar scripts y generar reportes de series de tiempo}
}

\newglossaryentry{rlang}{
  name={R},
  sort=r,
  description={Lenguaje de programación y entorno especializado en
               estadística y visualización de datos; base de muchos
               paquetes econométricos (p.\,ej.\ \textit{plm}, \textit{fixest})
               y punto de referencia para análisis reproducible}
}


\newglossaryentry{inteligenciaartificial}{
  name={inteligencia artificial (IA)},
  sort=inteligenciaartificial,
  description={Campo de la informática que desarrolla sistemas capaces
               de realizar tareas que normalmente requieren inteligencia
               humana; en ciencia de datos incluye aprendizaje automático
               y modelos generativos que hoy asisten la programación y
               el análisis econométrico}
}

% --------------------------------------------------------------------

% --- GLOSARIO | Cap. “Modelos de Regresión Lineal” ------------------

\newglossaryentry{ols}{
  name={mínimos cuadrados ordinarios (OLS)},
  sort=minimoscuadradosordinarios,
  description={Método que estima los coeficientes de la regresión
               lineal minimizando la suma de los residuos al cuadrado}
}

\newglossaryentry{blue}{
  name={BLUE},
  sort=blue,
  description={Siglas de \textit{Best Linear Unbiased Estimator};
               afirma que, bajo los supuestos de Gauss-Márkov, OLS es el
               estimador lineal insesgado con menor varianza}
}

\newglossaryentry{gauss-markov}{
  name={teorema de Gauss–Márkov},
  sort=gaussmarkov,
  description={Resultado que demuestra que, si se cumplen cinco
               supuestos (linealidad, muestreo aleatorio, ausencia de
               colinealidad perfecta, exogeneidad y homoscedasticidad),
               OLS es BLUE}
}

\newglossaryentry{multicolinealidad}{
  name={multicolinealidad},
  sort=multicolinealidad,
  description={Situación en la que dos o más regresores están altamente
               correlacionados, provocando estimaciones inestables y
               errores estándar inflados}
}

\newglossaryentry{homoscedasticidad}{
  name={homoscedasticidad},
  sort=homoscedasticidad,
  description={Propiedad según la cual la varianza de los errores es
               constante para todos los valores de los regresores}
}

\newglossaryentry{heteroscedasticidad}{
  name={heteroscedasticidad},
  sort=heteroscedasticidad,
  description={Caso opuesto a la homoscedasticidad; la varianza del
               error cambia con el nivel de los regresores, violando un
               supuesto clave de OLS}
}

\newglossaryentry{vif}{
  name={factor de inflación de la varianza (VIF)},
  sort=vif,
  description={Índice que cuantifica cuánto aumenta la varianza del
               coeficiente de un regresor debido a la multicolinealidad;
               valores superiores a 5–10 indican posible problema}
}

\newglossaryentry{r-cuadrado}{
  name={$R^{2}$},
  sort=rcuadrado,
  description={Medida de ajuste que indica la proporción de la
               variación de la variable dependiente explicada por el
               modelo; varía entre 0 y 1}
}

\newglossaryentry{estadistico-f}{
  name={estadístico $F$},
  sort=estadisticof,
  description={Contrasta la hipótesis nula de que todos los
               coeficientes (excepto la constante) son cero contra la
               alternativa de que al menos uno es distinto de cero}
}

\newglossaryentry{p-value}{
  name={\emph{p-value}},
  sort=pvalue,
  description={Probabilidad de obtener un estadístico tan extremo como
               el observado si la hipótesis nula fuera cierta; valores
               pequeños sugieren rechazar $H_{0}$}
}

\newglossaryentry{intervalo-confianza}{
  name={intervalo de confianza},
  sort=intervaloconfianza,
  description={Rango que, con una probabilidad pre-especificada
               (p.\,ej.\ 95 \%), contiene el verdadero valor del
               parámetro poblacional}
}

\newglossaryentry{error-estandar}{
  name={error estándar},
  sort=errorestandar,
  description={Estimación de la desviación típica de la distribución
               muestral de un coeficiente; mide la precisión de la
               estimación}
}

\newglossaryentry{residuales}{
  name={residuales},
  sort=residuales,
  description={Diferencias entre los valores observados y los
               predichos por el modelo; se emplean para diagnosticar los
               supuestos de la regresión}
}

\newglossaryentry{endogeneidad}{
  name={endogeneidad},
  sort=endogeneidad,
  description={Situación en la que un regresor está correlacionado con
               el término de error, violando el supuesto de exogeneidad
               y sesgando las estimaciones}
}

\newglossaryentry{media-condicional-cero}{
  name={media condicional cero},
  sort=mediacondicionalcero,
  description={Supuesto clave que exige que el valor esperado del error
               sea cero condicional a los regresores
               ($E[\varepsilon|X]=0$); garantiza exogeneidad}
}

% --- GLOSARIO | Cap. “Experimentos y Pruebas A/B” --------------------

\newglossaryentry{abtesting}{
  name={prueba A/B},
  sort=abtesting,
  description={Experimento controlado en el que se comparan dos variantes
               (A y B) que difieren en un solo elemento; la asignación
               aleatoria de usuarios permite estimar el efecto causal de
               la variante sobre un KPI (p.\,ej., tasa de conversión)}
}

\newglossaryentry{goldstandard}{
  name={gold standard experimental},
  sort=goldstandard,
  description={Expresión que alude al ensayo controlado aleatorio como
               referencia máxima para identificar efectos causales,
               pues elimina sesgos mediante aleatorización y control}
}

\newglossaryentry{aleatorizacion}{
  name={aleatorización},
  sort=aleatorizacion,
  description={Proceso de asignar unidades a tratamiento o control
               usando un mecanismo de azar; garantiza balance de
               variables observadas y no observadas, fundamento de la
               inferencia causal en experimentos}
}

\newglossaryentry{grupodetratamiento}{
  name={grupo de tratamiento},
  sort=grupodetratamiento,
  description={Conjunto de unidades que recibe la intervención o
               variante bajo estudio; se compara con el grupo de control
               para calcular el efecto causal}
}

\newglossaryentry{experimentoideal}{
  name={experimento ideal},
  sort=experimentoideal,
  description={Diseño hipotético sin restricciones prácticas ni éticas
               que ayuda a clarificar el mecanismo causal y la estrategia
               de identificación cuando un ensayo real no es factible}
}

% --- GLOSARIO | Cap. “El modelo de resultados potenciales” ----------

\newglossaryentry{resultados-potenciales}{
  name={resultados potenciales},
  sort=resultadospotenciales,
  description={Marco teórico que asocia a cada unidad dos posibles
               resultados: el que observaría si recibe el tratamiento
               ($Y_{1}$) y el que observaría si no lo recibe ($Y_{0}$).
               Sólo uno se observa en la práctica}
}

\newglossaryentry{contrafactual}{
  name={contrafactual},
  sort=contrafactual,
  description={Resultado potencial que \emph{no} se observa en la
               realidad; describe lo que habría ocurrido bajo un estado
               alternativo del mundo}
}

\newglossaryentry{variable-dummy}{
  name={variable indicadora (\textit{dummy})},
  sort=variabledummy,
  description={Variable binaria que toma valor 1 cuando la unidad
               pertenece a un grupo (p.\,ej.\ tratamiento) y 0 en caso
               contrario; útil para codificar categorías en modelos
               econométricos}
}

\newglossaryentry{grupo-tratamiento}{
  name={grupo de tratamiento},
  sort=grupotratamiento,
  description={Conjunto de unidades que reciben la intervención o
               condición cuyo efecto causal se desea estimar}
}


\newglossaryentry{sesgo-seleccion}{
  name={sesgo de selección},
  sort=sesgoseleccion,
  description={Distorsión que surge cuando la asignación al tratamiento
               no es aleatoria y está correlacionada con los resultados
               potenciales, de modo que la comparación directa entre
               grupos produce estimaciones sesgadas}
}

\newglossaryentry{att}{
  name={ATT},
  sort=att,
  description={\textit{Average Treatment on the Treated} (tratamiento
               promedio en los tratados): $E[Y_{1}-Y_{0}\mid D=1]$, el
               efecto causal medio para las unidades que efectivamente
               recibieron el tratamiento}
}

\newglossaryentry{rct}{
  name={ensayo controlado aleatorizado (RCT)},
  sort=rct,
  description={Diseño experimental en el que las unidades se asignan al
               tratamiento o al control mediante un procedimiento
               aleatorio, eliminando el sesgo de selección y
               permitiendo estimar efectos causales de manera creíble}
}


% --- GLOSARIO | Cap. “Series de tiempo” ------------------------------

\newglossaryentry{serie-tiempo}{
  name={serie de tiempo},
  sort=serietiempo,
  description={Secuencia ordenada $\{X_t\}$ de observaciones de una
               variable aleatoria a lo largo del tiempo}
}

\newglossaryentry{estacionariedad}{
  name={estacionariedad},
  sort=estacionariedad,
  description={Propiedad de un proceso estocástico cuyas
               características estadísticas (media, varianza y
               autocovarianza) permanecen constantes en el tiempo}
}

\newglossaryentry{caminata-aleatoria}{
  name={caminata aleatoria},
  sort=caminataaleatoria,
  description={Proceso $X_t=X_{t-1}+\varepsilon_t$ donde $\varepsilon_t$
               es ruido blanco; ejemplo clásico de serie no
               estacionaria}
}

\newglossaryentry{ruido-blanco}{
  name={ruido blanco},
  sort=ruidoblanco,
  description={Secuencia de innovaciones $\{\varepsilon_t\}$ con media
               cero, varianza constante finita y nula correlación serial}
}

\newglossaryentry{raiz-unitaria}{
  name={raíz unitaria},
  sort=raizunitaria,
  description={Raíz del polinomio autoregresivo igual a 1; su presencia
               indica no-estacionariedad}
}

\newglossaryentry{dickey-fuller}{
  name={prueba Dickey–Fuller (ADF)},
  sort=dickeyfuller,
  description={Contrastación estadística para la hipótesis de raíz
               unitaria; rechazar la hipótesis nula implica
               estacionariedad}
}

\newglossaryentry{diferencia}{
  name={diferencia de la serie},
  sort=diferencia,
  description={Transformación $\Delta X_t = X_t - X_{t-1}$ (o de orden
               superior) usada para inducir estacionariedad}
}

\newglossaryentry{acf}{
  name={ACF},
  sort=acf,
  description={\textit{Autocorrelation Function}: gráfico de las
               autocorrelaciones muestrales en distintos rezagos}
}

\newglossaryentry{pacf}{
  name={PACF},
  sort=pacf,
  description={\textit{Partial Autocorrelation Function}: muestra las
               autocorrelaciones \emph{condicionales}; útil para fijar
               el orden $p$ de un modelo AR}
}

\newglossaryentry{modelo-ar}{
  name={modelo autorregresivo AR($p$)},
  sort=modeloar,
  description={Modelo lineal que explica $Y_t$ a partir de sus $p$
               rezagos: $Y_t=\sum_{i=1}^{p}\phi_i Y_{t-i}+\varepsilon_t$}
}

\newglossaryentry{modelo-ma}{
  name={modelo de medias móviles MA($q$)},
  sort=modeloma,
  description={Modelo que expresa $Y_t$ como combinación lineal de los
               rezagos del error: $Y_t=\sum_{j=1}^{q}\theta_j
               \varepsilon_{t-j}+\varepsilon_t$}
}

\newglossaryentry{modelo-arma}{
  name={modelo ARMA($p,q$)},
  sort=modeloarma,
  description={Combina componentes AR($p$) y MA($q$) en un solo
               esquema: $Y_t=\sum_{i=1}^p\phi_i Y_{t-i}
               +\sum_{j=1}^q\theta_j\varepsilon_{t-j}+\varepsilon_t$}
}

\newglossaryentry{modelo-arima}{
  name={modelo ARIMA($p,d,q$)},
  sort=modeloarima,
  description={Extensión del ARMA a series no estacionarias mediante
               $d$ diferencias (Integración)}
}

\newglossaryentry{orden-rezago}{
  name={orden de rezago},
  sort=ordenrezago,
  description={Número de retardos ($p$ en AR, $q$ en MA) incorporados
               al modelo}
}

\newglossaryentry{aic}{
  name={AIC},
  sort=aic,
  description={\textit{Akaike Information Criterion}: medida de
               selección de modelos que penaliza la complejidad;
               menor AIC $\Rightarrow$ mejor ajuste/parcimonia}
}

\newglossaryentry{bic}{
  name={BIC},
  sort=bic,
  description={\textit{Bayesian Information Criterion}: versión más
               restrictiva que el AIC para comparar modelos}
}


% --- GLOSARIO | Cap. “Efectos fijos” ---------------------------------

\newglossaryentry{datos-panel}{
  name={datos en panel},
  sort=datospanel,
  description={Conjunto de observaciones $\{Y_{it}\}$ que combina
               dimensión transversal ($i$) y temporal ($t$) para los
               mismos individuos o unidades}
}

\newglossaryentry{efectos-fijos}{
  name={efectos fijos},
  sort=efectosfijos,
  description={Método de estimación que controla inobservables constantes
               dentro de cada unidad (o periodo) añadiendo una constante
               específica; se basa en la variación \textit{within}}
}

\newglossaryentry{within}{
  name={transformación within},
  sort=within,
  description={Centrado de cada variable respecto a su media por
               individuo: $\ddot X_{it}=X_{it}-\bar X_i$; elimina el
               término constante $U_i$ de los efectos fijos}
}

\newglossaryentry{efectos-fijos-dos-vias}{
  name={efectos fijos de dos vías},
  sort=efectosfijosdosvias,
  description={Modelo que introduce efectos fijos
               \emph{por entidad} ($U_i$) y
               \emph{por tiempo} ($U_t$) simultáneamente}
}

\newglossaryentry{poolabilidad}{
  name={prueba de poolabilidad},
  sort=poolabilidad,
  description={Contrastación (F-test) para decidir si un modelo
               agrupado sin efectos fijos es apropiado frente a un
               modelo con efectos fijos}
}

% --- GLOSARIO | Cap. “Diferencias-en-diferencias” ---------------------

\newglossaryentry{did}{
  name={diferencias en diferencias},
  text={DiD},
  sort=diferenciasendiferencias,
  description={Estrategia cuasi-experimental que identifica un efecto
               causal comparando la variación temporal de un grupo
               tratado con la de un grupo de control}
}


\newglossaryentry{sutva}{
  name={SUTVA},
  sort=sutva,
  description={\textit{Stable Unit Treatment Value Assumption}:
               no-interferencia entre unidades y único valor de
               tratamiento}
}

\newglossaryentry{no-anticipacion}{
  name={supuesto de no anticipación},
  sort=noanticipacion,
  description={Los resultados potenciales previos al tratamiento no
               se ven afectados por la futura asignación de éste}
}

\newglossaryentry{superposicion-fuerte}{
  name={superposición fuerte},
  sort=superposicionfuerte,
  description={Para cada estrato de covariables existe probabilidad
               positiva de pertenecer a cualquier grupo de tratamiento}
}

\newglossaryentry{tendencias-paralelas}{
  name={tendencias paralelas},
  sort=tendenciasparalelas,
  description={Supuesto clave de DiD: en ausencia de tratamiento, la
               evolución promedio del resultado sería igual en ambos
               grupos}
}

\newglossaryentry{interaccion}{
  name={término de interacción},
  sort=interaccion,
  description={Producto de dos variables (p.\,ej.\ $D_i\times T_t$)
               que captura un efecto conjunto; en DiD su coeficiente es
               el estimador causal}
}

% --- GLOSARIO | Cap. “Investigación de mercados e IA” ---------------

\newglossaryentry{investigacion-mercados}{
  name={investigación de mercado},
  sort=investigacionmercado,
  description={Proceso sistemático de obtención, organización y análisis
               de información relevante sobre consumidores, competidores
               y entorno, para apoyar la toma de decisiones de negocio}
}

\newglossaryentry{focus-group}{
  name={grupo de enfoque},
  text={focus-group},
  sort=focusgroup,
  description={Técnica cualitativa que reúne a varias personas bajo la
               guía de un moderador para explorar actitudes, opiniones y
               motivaciones acerca de un producto o idea}
}

\newglossaryentry{mom-test}{
  name={\textit{Mom Test}},
  sort=momtest,
  description={Metodología de Rob Fitzpatrick para formular preguntas
               que obliguen al entrevistado (incluso tu madre) a dar
               retroalimentación honesta basada en hechos y no en
               halagos}
}

\newglossaryentry{escala-likert}{
  name={escala de Likert},
  sort=likert,
  description={Formato de pregunta cerrado en el que el encuestado
               indica su nivel de acuerdo con una afirmación en una
               gradación ordinal (p.\,ej.\ 1 – 5 o 1 – 7)}
}

% --- GLOSARIO | Cap. “Iterar para crear valor” -----------------------

\newglossaryentry{mvp}{
  name={MVP},
  description={Sigla de \emph{mínimo producto viable}. Versión más
               sencilla de un producto o servicio que permite poner a
               prueba la propuesta de valor con el mínimo tiempo y
               coste posibles, obteniendo retroalimentación temprana
               del mercado},
  text={MVP},
  sort=mvp
}

\newglossaryentry{lean-startup}{
  name={\textit{Lean Startup}},
  sort=leanstartup,
  description={Metodología de creación de negocios de Eric Ries
               basada en ciclos rápidos “construir–medir–aprender”,
               experimentación continua y aprendizaje validado para
               reducir riesgo e inversión}
}

\newglossaryentry{vanity-metric}{
  name={métrica de vanidad},
  sort=vanitymetric,
  description={Indicador que “luce” bien (p.\,ej.\ followers, likes) pero
               no orienta la toma de decisiones porque no refleja
               directamente la salud o el crecimiento económico del
               negocio}
}

\newglossaryentry{cac}{
  name={CAC},
  description={\emph{Customer Acquisition Cost}. Costo total de captar
               un nuevo cliente, calculado dividiendo la inversión en
               marketing y ventas entre el número de clientes
               adquiridos en un periodo},
  text={CAC},
  sort=cac
}

\newglossaryentry{hipotesis-mercados-eficientes}{
  name={hipótesis de los mercados eficientes},
  sort=mercadoseficientes,
  description={Proposición de que toda la información relevante se
               incorpora instantáneamente a los precios de mercado; de
               ser cierta, las oportunidades de ganancia “fáciles” se
               eliminan de forma inmediata}
}

